\documentclass[a4paper,10pt]{llncs}

\usepackage{graphicx}
\usepackage{multicol}
\usepackage{amsbsy,amscd,amsfonts,amssymb,amstext,amsmath,latexsym,theorem}

\pagestyle{plain}
\bibliographystyle{alpha}

%%%%%%%%%%%%%%%%%%%%%%%%%%%%%%%%%%%%%%%%%%%%%%%%%%%%%%%%%%%%%%%%%%%%%

\begin{document}

\title{{\normalsize Seminar: Formal Specification} \\[1ex]
  Security of Multithreaded Programs by Compilation\cite{Barthe07}}
\author{Pascal Wittmann, Advisor: Artem Starostin}
\institute{TU Darmstadt}

\maketitle

%%%%%%%%%%%%%%%%%%%%%%%%%%%%%%%%%%%%%%%%%%%%%%%%%%%%%%%%%%%%%%%%%%%%%

{\bf Here comes the abstract!}

\section{Motivation}
\label{sec:motivation}
Barthe, Rezk, Russo and Sabelfeld introduced with
their paper \cite{Barthe07} a framework to guarantee
noninterference in multithreaded programs at byte-code
level. Noninterference is a security property which
says that a program does not leak sensitive information
to an adversary. Since more and more (mobile) devices handle
an increasing number of tasks, multithreading is widely used
to prevent lock ups (when e.g. establishing a 
network-connection). But often programs that are secure when they
are executed sequentially, leak information when they are
composed in parallel. This leads to an attractive channel of
information leakage.

The idea of Barthe et. al. was to close this channel by compiling
type information into the byte-code, which cause the scheduler to
treat the secure parts in a different way. Since many mobile
platforms already use some byte-code (e.g. Android a modification
of the JVM byte-code).

\section{Introduction to research area}
\label{sec:introduction}
\section{Discussion of a solution}
\label{sec:discussion}
\section{Related Work}
\label{sec:relatedwork}
\section{Conclusion \& Outlook}
\label{sec:conclusion}

\bibliography{bibliography}

\end{document}
