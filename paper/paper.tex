\documentclass[a4paper,10pt]{llncs}

\usepackage{graphicx}
\usepackage{multicol}
\usepackage{amsbsy,amscd,amsfonts,amssymb,amstext,amsmath,latexsym,theorem}

\pagestyle{plain}
\bibliographystyle{alpha}

%%%%%%%%%%%%%%%%%%%%%%%%%%%%%%%%%%%%%%%%%%%%%%%%%%%%%%%%%%%%%%%%%%%%%

\begin{document}

\title{{\normalsize Seminar: Formal Specification} \\[1ex]
  Security of Multithreaded Programs by Compilation\cite{Barthe07}}
\author{Pascal Wittmann, Advisor: Artem Starostin}
\institute{TU Darmstadt}

\maketitle

%%%%%%%%%%%%%%%%%%%%%%%%%%%%%%%%%%%%%%%%%%%%%%%%%%%%%%%%%%%%%%%%%%%%%

{\bf Here comes the abstract!}

\section{Motivation}
\label{sec:motivation}
Barthe, Rezk, Russo and Sabelfeld introduced with
their paper \cite{Barthe07} a framework to guarantee
noninterference in multithreaded programs at byte-code
level. Noninterference is a security property which
says that a program does not leak sensitive information
to an adversary. Since more and more (mobile) devices handle
an increasing number of tasks, multithreading is widely used
to prevent lock ups (when e.g. establishing a 
network-connection). But often programs that are secure when they
are executed sequentially, leak information when they are
composed in parallel. This leads to an attractive channel of
information leakage.

The idea of Barthe et. al. was to close this channel by compiling
type information into the byte-code, which cause the scheduler to
treat the secure parts in a different way. The fact that many mobile
platforms already use some byte-code (e.g. Android uses a modification
of the JVM byte-code) may have been a reason for the decision to use
type-annotated byte-code.

\section{Introduction to research area}
\label{sec:introduction}
\section{Summary of the article}
\label{sec:discussion}
In this summary I will follow mainly the structure of the original
paper\cite{Barthe07} which is as follows. After the introduction the basic
terms and definitions for a multithreaded programs and the scheduler are
laid. After that the notion of security we want to achieve is presented.
Along with this a skeleton of a type system is described, which ensures
that a program typable in this type system is secure w.r.t the notion of
security. The proof that this holds is sketched in the following section.
In the last section the (by now) abstract framework is instantiated with a
concrete example.

\subsection{Syntax and Semantics of multithreaded programs}
\label{sec:syntaxsemantics}
A program is viewed as an abstract thing, which consists of a set of
program points $\mathcal{P}$ with a distinguished entry ($1$) and exit
(\texttt{exit}) point and a function that maps program points to
instructions.

These instructions are not further specified, but contain an instruction
to create a new thread (\texttt{start \textit{pc}} where pc is the start
instruction of the new thread).

Further there is a relation $\mapsto$ that describes possible successor
instructions. \texttt{exit} is the only instructions with no successor
and \texttt{start \textit{pc}} may only have a single successor (the
following program point).

The next thing introduced are the security levels. We assume the attacker
'is' a level $k$. From this assumption we can reduce every set of levels
w.l.o.g into $\{low, high\}$, where $low < high$, by mapping elements
that are no more sensitive than $k$ to $low$ and all other elements
including incomparable ones to $high$.

\subsection{Type system}
\label{sec:typesystem}
\subsection{Soundness}
\label{sec:soundness}
\subsection{Instantiation}
\label{sec:instantiation}

\section{Related Work}
\label{sec:relatedwork}
\section{Conclusion \& Outlook}
\label{sec:conclusion}

\bibliography{bibliography}

\end{document}
